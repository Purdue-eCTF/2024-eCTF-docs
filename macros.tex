%%%%%%%%%%%%%%%%%%%%%%%%%%%%%%%%%%%%%%%%%%%%%%%%%%%%%%%%%%%%%%%%%%%%
% Paper-specific macros go here...
%%%%%%%%%%%%%%%%%%%%%%%%%%%%%%%%%%%%%%%%%%%%%%%%%%%%%%%%%%%%%%%%%%%%

\newcommand{\enc}{\ensuremath{\mathsf{Enc}}\xspace}
\newcommand{\dec}{\ensuremath{\mathsf{Dec}}\xspace}
\renewcommand{\H}{\ensuremath{\mathsf{H}}\xspace}

\newcommand{\seed}{\ensuremath{S}\xspace}
\newcommand{\prng}{\ensuremath{\mathsf{PRNG}}\xspace}
%\newcommand{\hmac}{\ensuremath{\mathsf{HMAC}}\xspace}

% ectf{...}
\newcommand{\flag}[1]{\texttt{ectf\{#1\}}}
\newcommand{\pin}{\ensuremath{\mathsf{PIN}}\xspace}
\newcommand{\tok}{\ensuremath{\mathsf{token}}\xspace}
\newcommand{\compid}{\ensuremath{\mathsf{ID}}\xspace}

%\SetKw{Continue}{continue}

\NewDocumentCommand{\mysendmessageleft}{m}{%
  \sendmessage{<-,inner sep=0pt}{
    topstyle={inner sep=2pt},
    top=\smash[t]{$\scriptstyle#1$},
  }%
}
\NewDocumentCommand{\mysendmessageright}{m}{%
  \sendmessage{->,inner sep=0pt}{
    topstyle={inner sep=2pt},
    top=\smash[t]{$\scriptstyle#1$},
  }%
}

%%%%%%%%%%%%%%%%%%%%%%%%%%%%%%%%%%%%%%%%%%%%%%%%%%%%%%%%%%%%%%%%%%%%

\newcommand{\cmark}{\ding{51}}%
\newcommand{\xmark}{\ding{55}}%

%%%%%%%%%%%%%%%%%%%
% Math macros     %
%%%%%%%%%%%%%%%%%%%

%%% General Arithmetic
\newcommand{\abs}[1]{\ensuremath{\left| #1 \right|}\xspace}

%%% Linear Algebra
\renewcommand{\vec}[1]{\ensuremath{\mathbf{#1}}\xspace}
%\renewcommand{\vec}[1]{\ensuremath{\overrightarrow{#1}}}

%%% Groups, Fields, and Rings
\newcommand{\group}{\ensuremath{\mathbb{G}}\xspace}
%\newcommand{\G}{\group}
\newcommand{\nat}{\ensuremath{\mathbb{N}}\xspace}
\newcommand{\naturals}{\nat}
\newcommand{\N}{\nat}

\newcommand{\ints}{\ensuremath{\mathbb{Z}}\xspace}
\newcommand{\integers}{\ints}
\newcommand{\Z}{\ints}
\newcommand{\Zmod}[1]{\ensuremath{\Z_{#1}}\xspace}

\newcommand{\rat}{\ensuremath{\mathbb{Q}}\xspace}
\newcommand{\rationals}{\rat}
\newcommand{\Q}{\rat}

\newcommand{\real}{\ensuremath{\mathbb{R}}\xspace}
\newcommand{\reals}{\real}
\newcommand{\R}{\real}

\newcommand{\complex}{\ensuremath{\mathbb{C}}\xspace}
\newcommand{\C}{\complex}

\newcommand{\field}{\ensuremath{\mathbb{F}}\xspace}
\newcommand{\F}{\field}


%%% Set notation
\newcommand{\powerset}[1]{\ensuremath{\mathcal{P}(#1)}\xspace}
\newcommand{\set}[1]{\ensuremath{\left\{ #1 \right\}}\xspace}
%%% NOTE: Change if you prefer colon or vertical bar (\mid) for suchthat
%\newcommand{\suchthat}{\mathbin{:}}
\newcommand{\suchthat}{\mathbin{\mid}}
\newcommand{\given}{\suchthat}
%%% Prevents overloading strikethrough macro if the document imports soul package
\ifdefined\st\else\newcommand{\st}{\suchthat}\fi
\newcommand{\cset}[2]{\set{#1 \suchthat #2}}  % a conditional notation to define sets
\newcommand{\lset}[2]{\set{#1,\ldots,#2}} % set {from,...,to}
\renewcommand{\complement}[1]{\ensuremath{\overline{#1}}}
\newcommand{\univset}{\ensuremath{U}\xspace}

%%% Custom functions

%%% Custom operators
\newcommand{\bigcdot}{\mathbin{\bullet}}
\newcommand{\iso}{\mathbin{\cong}}
\newcommand{\grad}{\ensuremath{\nabla}}
%\newcommand{\possemidef}{\mathbin{\succeq}}
%\newcommand{\posdef}{\mathbin{\succ}}
%%% Custom equivalence Relations
\renewcommand{\approx}{\raise.17ex\hbox{\ensuremath{\scriptstyle\sim}}}
%%% Relations-checking operators
\newcommand{\iseq}{\overset{?}{=}}
\newcommand{\islt}{\overset{?}{<}}
\newcommand{\isgt}{\overset{?}{>}}
\newcommand{\isleq}{\overset{?}{\leq}}
\newcommand{\isgeq}{\overset{?}{\geq}}

%%% Statistics
% NOTE: default for \Pr is Pr
%\DeclareMathOperator{\Prob}{P}
% NOTE: Sometimes people like to use regular E instead of Blackboard Bold
\DeclareMathOperator{\Exp}{\mathbb{E}}
%\DeclareMathOperator{\Exp}{E}
% NOTE: \Pr already exists and is normalscript math operator "Pr". I don't like that one.
\DeclareMathOperator{\Prob}{\mathbb{P}}
%\DeclareMathOperator{\Prob}{P}
\DeclareMathOperator{\Var}{Var}

%%%%%%%%%%%%%%%%%%%
% CS Macros       %
%%%%%%%%%%%%%%%%%%%
%%% General
\newcommand{\alg}[1]{\ensuremath{\mathsf{#1}}\xspace}
\newcommand{\cmd}[1]{\texttt{#1}\xspace}
\newcommand{\var}[1]{\ensuremath{\mathsf{#1}}\xspace}
%\newcommand{\true}{\var{T}}
%\newcommand{\false}{\var{F}}
%\newcommand{\boolset}{\set{\true,\false}}
\newcommand{\bitset}{\set{0,1}}
\newcommand{\boolset}{\bitset}
\newcommand{\bitsset}[1][*]{\ensuremath{\bitset^{#1}}\xspace}
\renewcommand{\implies}{\mathbin{\longrightarrow}}
\newcommand{\addeq}{\ensuremath{+=}\xspace}
\newcommand{\subeq}{\ensuremath{-=}\xspace}
\newcommand{\defeq}{\coloneqq}
\newcommand{\eqdef}{=:}
\newcommand{\nil}{\ensuremath{\var{nil}}\xspace}
\newcommand{\error}{\ensuremath{\bot}\xspace}
\newcommand{\varlist}[1]{\ensuremath{\overline{#1}}\xspace}

%%% Computation theory
\newcommand{\blanksymbol}{\ensuremath{%
  \ooalign{%
    \relax\cr%
    \noalign{\vskip.2ex}%
    \textvisiblespace\cr%
    \noalign{\vskip-.2ex}%
    \hphantom{~}\cr%   Width of a space character
    \vphantom{Xg}\cr}%  Height of X, depth of g
  }\xspace%
}
\newcommand{\desc}[1]{\ensuremath{\left\langle #1 \right\rangle}\xspace}
%%% Formal Syntax + Grammars + Languages
\newcommand{\derives}{\mathbin{\rightarrow}}
\newcommand{\reduces}{\derives}
\newcommand{\derivesto}[1][*]{\mathbin{\overset{#1}{\rightarrow}}}
\newcommand{\reducesto}{\derivesto}
%%% From: https://tex.stackexchange.com/questions/41357/replace-cmtt10-char32-visible-cup-space-with-something-more-gentle/41445#41445
\newcommand{\emptystring}{\ensuremath{\epsilon}\xspace}
\newcommand{\emptysymbol}{\emptystring}
\newcommand{\any}[1]{{#1}^*}
\newcommand{\kleenestar}[1]{\any{#1}}
\newcommand{\strconcat}{\mathbin{\Vert}}
\newcommand{\concat}{\strconcat}
%%% Cryptography
\newcommand{\adversary}{\ensuremath{\mathcal{A}}\xspace}

%%%%%%%%%%%%%%%%%%%
% Theorems etc.   %
%%%%%%%%%%%%%%%%%%%

%\renewenvironment{proof} {\vspace{0.5em} \noindent \underline{\textbf{Pf}}:} {\hfill \qedsymbol \vspace{0.5em}}      % Default proof, but with _*Pf:*_ instead
\newenvironment{theoremproof} {\noindent \textit{Proof \thetheorem.}} {\hfill \qedsymbol \vspace{0.5em}}      % Default proof, numbered by theorem
\newenvironment{corollaryproof} {\noindent \textit{Proof \thecorollary.}} {\hfill \qedsymbol \vspace{0.5em}}      % Default proof, numbered by corollary
\newenvironment{lemmaproof} {\noindent \textit{Proof \thelemma.}} {\hfill \qedsymbol \vspace{0.5em}}      % Default proof, numbered by lemma
\newenvironment{proofsketch} {\noindent \textit{Proof (Sketch).}} {\vspace{0.5em}}      

%%% Default proofs, but as an informal description
\ifdefined\definition\else\newtheorem{definition}{Definition}\fi
\newtheorem*{definition*}{Definition}
%%% NOTE: use "problem" if HW, "theorem" if generic, or "section" if paper
\ifdefined\theorem\else\newtheorem{theorem}{Theorem}[section]\fi
\newtheorem*{theorem*}{Theorem}
%%% NOTE: use "problem" if HW, "theorem" if generic, or "section" if paper
\ifdefined\assumption\else\newtheorem{assumption}{Assumption}[section]\fi
\newtheorem*{assumption*}{Assumption}
\ifdefined\corollary\else\newtheorem{corollary}{Corollary}[theorem]\fi
\newtheorem*{corollary*}{Corollary}
\ifdefined\lemma\else\newtheorem{lemma}{Lemma}[theorem]\fi
\newtheorem*{lemma*}{Lemma}
%%% NOTE: use "problem" if HW, "theorem" if generic, or "section" if paper
\ifdefined\claim\else\newtheorem{claim}{Claim}[section]\fi
\newtheorem*{claim*}{Claim}
\ifdefined\observation\else\newtheorem*{observation}{Observation}\fi
\ifdefined\proposition\else\newtheorem*{proposition}{Proposition}\fi
\ifdefined\fact\else\newtheorem*{fact}{Fact}\fi
\ifdefined\remark\else\newtheorem*{remark}{Remark}\fi

% Modifies the enumerate* to use 1) 2) 3) by default
\newlist{enumerate*}{enumerate*}{1}
\setlist[enumerate*]{label=\arabic*)}

%%%%%%%%%%%%%%%%%%%
% Text shortcuts  %
%%%%%%%%%%%%%%%%%%%
\renewcommand{\th}{\textsuperscript{th}\xspace}
\newcommand{\wrt}{\xspace{with respect to}\xspace}
\newcommand{\Wrt}{\xspace{With respect to}\xspace}
%%% NOTE: \wlog is a logging command and can't be redefined
\newcommand{\WLOG}{\xspace{without loss of generality}\xspace}
\newcommand{\Wlog}{\xspace{Without loss of generality}\xspace}
