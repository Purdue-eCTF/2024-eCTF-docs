\section{Threat Model}
\label{sec:threat-model}

\subsection{Devices}

\paragraph{Component.} A Component performs functionality that is necessary for a medical device to function as intended. The functionality of these Components are arbitrary and irrelevant to this competition, but might represent something like e.g. an insulin pump's actuator.

\paragraph{Application Processor.} The Application Processor (AP, for short) functions as a microcontroller for the Medical Device's Components, with Host tools controlling it via UART.

\paragraph{Medical Device.} An AP together with its Component(s). A MISC medical device must support the following functionality:
\begin{enumerate*}[label={(\arabic*)}]
    \item list all provisioned and connected Components;
    \item replace a provisioned Component;
    \item attest to confidential Component data;
    \item boot with valid Components; and
    \item send authenticated data between the AP and Components.% after booting.
\end{enumerate*}

\subsection{Users}

\paragraph{Attacker.} An Attacker is a malicious party who is attempting to violate the security properties of a medical device -- install counterfeit Components, compromise the build process, leak confidential data, etc. The Attacker either has physical or supply chain access.

\paragraph{Manufacturer.} The Manufacturer is a trusted party which generates global secrets, builds AP firmware, and provisions the AP with data which helps authenticate Components on the medical device. We emphasize that, unlike some real-world supply chain scenarios, it remains outside the competition's threat model for an Attacker to compromise the global secrets.

\paragraph{Component Facility.} The Component Facility uses secure global secrets generated by the Manufacturer to build arbitrary medical Component firmware for a medical device. In some scenarios, the Attestation PIN which allows authorized Users access to a Component's confidential Attestation Data (e.g. patient data) might be compromised by the Attacker.

\paragraph{Repair Technician.} The Repair Technician is responsible for replacing ``damaged'' Components (i.e., not visible to AP) with newly provisioned ones. In some scenarios, the replacement token for validating the Component swap might be compromised by the Attacker.

\paragraph{Operator.} The Operator (e.g., a patient or technician) runs the Manufacturer's provided host tools on a general-purpose PC, sending commands to the AP over a UART serial interface to operate the MISC medical device. This simulates real-world device controls.


%\parhead{Cars} In this scenario, a car is built with a list of available features, with the user having a subset of its features packaged. With these, a user who unlocks the car with their fob (ideally paired) should be able to send packaged features for the car to recognize and enable.

%\parhead{Fobs} A user can have fobs which are either \emph{unpaired} or \emph{paired} or with a particular car (for short, UPFOB or PFOB resp.). Every car comes with an initial PFOB which can subsequently be used to pair UPFOBs. Newly-paired fobs should operate identically to the initially-paired fob as well (i.e. it should be able to pair new fobs and unlock the car).

%We will develop the PARED protocols to run on these microcontrollers, and their binaries will be encrypted by MITRE's secure bootloader.

% \iflong
% \begin{figure}[Ht!]
% 	\begin{center}
% 		%\includegraphics[width=0.8\linewidth]{media/cars.png}
% 	\end{center}
% 	\caption{A list of cars provided to attackers and access allowed to each}
% 	\label{fig:cars}
% \end{figure}
% \fi

\subsection{Security Requirements}
\label{sec:SR}

\begin{enumerate}[label={\textbf{(SR\arabic*)}}, leftmargin=0.55in]
\item The AP must boot only after it confirms the validity of all necessary Components.

\item The Component must confirm the validity of the AP. Components must not boot before being commanded to do so by a valid AP. 

\item The Attestation PIN and Replacement Token must be kept secret to prevent attackers from accessing Attestation Data or booting with counterfeit Components.

\item The Attestation Data must remain confidential at all times, and should only be revealed to a valid AP after the user supplies a correct Attestation PIN.

\item The post-boot MISC communications must be secure against any tampering, forgery, and replay attacks; that is, message authenticity and integrity must be preserved. 
\end{enumerate}

For more information, refer to ``Security Requirements'' in the documentation \cite{eCTFOfficial}.

\subsection{Attack Scenarios}
\label{sec:AS}

\begin{enumerate}[label={\textbf{(AS\arabic*)}}, leftmargin=0.55in]
\item \textbf{Operational Device}: Attackers do not have the PIN or replacement token, but have physical access to a fully functional and valid system (i.e., AP and 2 Components).

\item \textbf{Damaged Device}: Same as AS1, but attackers only have 1 working Component.

\item \textbf{Supply Chain Poisoning}: Attackers obtain the Attestation PIN and Replace Token, and can load counterfeit firmware onto the board to extract data and boot.

\item \textbf{Black Box}: Attackers reverse-engineer 1 Component to extract data and boot.
\end{enumerate}

For more information, refer to the ``Attack Phase Flags and Scenarios'' section of the documentation \cite{eCTFOfficial}. \iflong For more details about how our proposed design addresses these threats and maps Attack Scenarios to Security Requirements, see the table in \Cref{app:table}.\else We used a variant of the MITRE ATT\&CK framework to assess the most important exploits related to the above.\fi

\subsection{Deployments}

Each team's medical devices are built and distributed to attacking teams in 3 ``Deployments'':
\begin{enumerate}[label={\textbf{(D\arabic*)}}, leftmargin=0.55in]
    \item AP1 provisioned for Components A \& B, AP2 provisioned for Components C \& D, and Components A,B, \& C;
    \item Counterfeit Component X; and
    \item AP3 provisioned for Components A \& B, Component A, and Replace Token \& Attestation PIN
\end{enumerate}
We assume the MITRE Organizers' bootloader and flags are securely stored on the boards, and regardless (as per eCTF rules) will not be directly attacked by other teams \cite{eCTFOfficial}.

\subsection{Attack Flags}

In summary, the table below describes each flag that an attacker is trying to gain \cite{eCTFOfficial}:

\begin{center}
\begin{tabular}{|l||c|c|l|}
    \hline
    \thead{Flag Name}
        & \thead{Deployment}
        & \thead{Scenario}
        & \thead{Flag Format}
        %& \thead{Description}
    \\ \hline\hline
    Operational Pin Extract
        & D1
        & (AS1, SR3)
        & \texttt{ectf\{pinextract\_*\}}
        %& \makecell{Return confidential attestation data from the operational device.}
    \\ \hline
    Operational Pump Swap
        & D1
        & (AS1, SR5)
        & \texttt{ectf\{pumpswap\_*\}}
        %& \makecell{Cause the operational insulin pump to dispense a dangerous level of insulin.}
    \\ \hline
    Damaged Boot
        & D1
        & (AS2, SR1)
        & \texttt{ectf\{damagedboot\_*\}}
        %& \makecell{Cause a damaged board missing a component to boot.}
    \\ \hline
    Black Box Boot
        & D2
        & (AS4, SR2)
        & \texttt{ectf\{blackboxboot\_*\}}
        %& \makecell{Cause a black box component to boot.}
    \\ \hline
    Black Box Extract
        & D2
        & (AS4, SR4)
        & \texttt{ectf\{blaxkboxextract\_*\}}
        %& \makecell{Extract confidential data from a black box component.}
    \\ \hline
    Supply Chain Boot
        & D3
        & (AS3, SR1)
        & \texttt{ectf\{supplychainboot\_*\}}
        %& \makecell{Cause a board missing a component, with a known PIN, and no physical access to boot.}
    \\ \hline
    Supply Chain Extract
        & D3
        & (AS3, SR4)
        & \texttt{ectf\{supplychainextract\_*\}}
        %& \makecell{Extract confidential data from a valid component connected to an AP with a known PIN and no physical access.}
    \\ \hline
\end{tabular}
\end{center}

%For convenience, we re-iterate the security requirements (SRs) which our secured PARED protocol satisfies.

%\parhead{SR1} A car should only unlock and start when the user has an authentic fob that is paired with the car.
%\iflong
%\begin{itemize}
%	\item Unlock Car 1 with an UPFOB: \flag{newcar\_*}
%\end{itemize}
%\fi

%\parhead{SR2} Revoking an attacker's physical access to a fob should also revoke their ability to unlock the associated car.
%\iflong
%\begin{itemize}
%	\item Protected PFOB access, disable at time $t$, then Protected UPFOB access
%	\item Unlock Car 2 with temporary PFOB access: \flag{tempfobaccess\_*}
%\end{itemize}
%\fi

%\parhead{SR3} Observing the communications between a fob and a car while unlocking should not allow an attacker to unlock the car in the future.
%\iflong
%\begin{itemize}
%	\item Protected UPFOB access
%	\item Recorded 3 unlock transactions between a paired fob and a car
%	\item Unlock Car 3 with passive replay attacks: \flag{passiveunlock\_*}
%\end{itemize}
%\fi

%\parhead{SR4} Having an unpaired fob should not allow an attacker to unlock a car without the corresponding paired fob and pairing PIN.
%\iflong
%\begin{itemize}
%	\item Protected UPFOB access and leaked pairing PIN for Car 4
%	\item Unlock Car 4 with just the PIN: \flag{leakedpin\_*}
%	\item Protected PFOB, UPFOB access for Car 5
%	\item Extract the pairing PIN from Car 5: \flag{pinextract\_*}
%\end{itemize}
%\fi

%\parhead{SR5} A car owner should not be able to add new features to a fob that did not get packaged by the manufacturer.

%\parhead{SR6} Access to a feature packaged for one car should not allow an attacker to enable the same feature on another car.
%\iflong
%\begin{itemize}
%	\item Protected PFOB, UPFOB access for Car 5
%	\item Unlocking Car 5 is out of scope
%	\item Packaged Feature 2 for Cars 0-4
%	\item Enable Feature 2 for Car 5: \flag{feature2\_*} 
%\end{itemize}
%\fi

%See \Cref{sec:build}--\ref{sec:unlock} for a description of how our modified PARED protocol satisfies each SR. With the threat model defined, we now describe the cryptographic primitives which provide assurances of security in this scenario.

%\iflong
%\textit{NOTE: Car 0 contains no flags, but \emph{does} provide the unprotected images for the car and paired fob. %, as well as a protected unpaired fob, paired fob, car, and PIN.
%With these capabilities, we expect that attackers can analyze our implementation for vulnerabilities.}
%\fi