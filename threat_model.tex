\section{Threat Model}
\label{sec:threat-model}

\subsection{Devices}

\paragraph{Component.} A Component performs functionality that is necessary for a medical device to function as intended. The functionality of these Components are arbitrary and irrelevant to this competition, but might represent something like e.g. an insulin pump's actuator.

\paragraph{Application Processor.} The Application Processor (AP, for short) functions as a microcontroller for the Medical Device's Components, with Host tools controlling it via UART.

\paragraph{Medical Device.} An AP together with its validated Component(s). A MISC medical device must support the following functionality:
\begin{enumerate*}[label={(\arabic*)}]
    \item list all provisioned and connected Components;
    \item replace a provisioned Component;
    \item attest to confidential data on Component;
    \item boot the device with validated and working Components; and
    \item send authenticated data between AP and Components after booting.
\end{enumerate*}

\subsection{Users}

\paragraph{Attacker.} An Attacker is a malicious party who is attempting to violate the security properties of a medical device -- install non-validated Components, compromise the build process, leak confidential data, etc. The Attacker either has physical or supply chain access.

\paragraph{Manufacturer.} The Manufacturer is a trusted party which generates global secrets, builds AP firmware, and provisions the AP with data which helps authenticate Components on the medical device. We emphasize that, unlike some real-world supply chain scenarios, it remains outside the competition's threat model for an Attacker to compromise the global secrets.

\paragraph{Component Facility.} The Component Facility uses secure global secrets generated by the Manufacturer to build arbitrary medical Component firmware for a medical device. In some scenarios, the Attestation PIN which allows authorized Users access to a Component's confidential Attestation Data (e.g. Patient's PII) might be compromised by the Attacker.

\paragraph{Repair Technician.} The Repair Technician is responsible for replacing ``damaged'' Components (i.e., not visible to AP) with newly provisioned ones. In some scenarios, the replacement token for validating the Component swap might be compromised by the Attacker.

\paragraph{Host / Operator.} The `Operator' (e.g., a patient or technician) runs the Manufacturer's host tools on a general-purpose PC, sending commands to the AP over a UART serial interface to operate the MISC medical device. This simulates real-world device controls.


%\parhead{Cars} In this scenario, a car is built with a list of available features, with the user having a subset of its features packaged. With these, a user who unlocks the car with their fob (ideally paired) should be able to send packaged features for the car to recognize and enable.

%\parhead{Fobs} A user can have fobs which are either \emph{unpaired} or \emph{paired} or with a particular car (for short, UPFOB or PFOB resp.). Every car comes with an initial PFOB which can subsequently be used to pair UPFOBs. Newly-paired fobs should operate identically to the initially-paired fob as well (i.e. it should be able to pair new fobs and unlock the car).

%We will develop the PARED protocols to run on these microcontrollers, and their binaries will be encrypted by MITRE's secure bootloader.

% \iflong
% \begin{figure}[Ht!]
% 	\begin{center}
% 		%\includegraphics[width=0.8\linewidth]{media/cars.png}
% 	\end{center}
% 	\caption{A list of cars provided to attackers and access allowed to each}
% 	\label{fig:cars}
% \end{figure}
% \fi

\subsection{Security Requirements}
\label{sec:SR}

\begin{enumerate}[label={\textbf{(SR\arabic*)}}, leftmargin=0.55in]
\item The Application Processor may boot only when it confirms the presence of every required Component as well as its validity.

\item The Application Processor, that was already checked to be a valid one, must first confirm the device integrity before allowing its Components to boot. Components must not boot before being properly ordered by a valid Application Processor. 

\item Attestation PIN and Replacement Token are secret information that must remain confidential to prevent unauthorized parties from gaining access to the device.

\item The Attestation Data of Components are secret information and need to remain confidential at all times as it determines the validity of the Component. 

\item The post-boot MISC communications should be secure against any tampering, duplication, and forgery; that is, message authenticity and integrity must be preserved. 
\end{enumerate}

For more information, refer to the `Security Requirement' section of the documentation \cite{eCTFOfficial}.

\subsection{Attack Scenarios}
\label{sec:AS}

\begin{enumerate}[label={\textbf{(AS\arabic*)}}, leftmargin=0.55in]
\item \textbf{Operational Device}: Attackers do not have the PIN or the replacement token, but have physical access to a fully functional system (i.e., AP and 2 Components).

\item \textbf{Damaged Device}: Attackers have same access as AS1, but only have 1 Component.

\item \textbf{Supply Chain Poisoning}: Attackers obtain the attestation PIN and replacement token from a technician, trying to boot a counterfeit Component and extract data.

\item \textbf{Black Box}: Attackers reverse-engineer 1 Component to boot and extract data.
\end{enumerate}

For more information, refer to the `Attack Phase Flags and Scenarios' section of the documentation \cite{eCTFOfficial}. For more details about how our proposed design addresses these threats and maps Attack Scenarios to Security Requirements, see the table in \Cref{app:table}. 


%For convenience, we re-iterate the security requirements (SRs) which our secured PARED protocol satisfies.

%\parhead{SR1} A car should only unlock and start when the user has an authentic fob that is paired with the car.
%\iflong
%\begin{itemize}
%	\item Unlock Car 1 with an UPFOB: \flag{newcar\_*}
%\end{itemize}
%\fi

%\parhead{SR2} Revoking an attacker's physical access to a fob should also revoke their ability to unlock the associated car.
%\iflong
%\begin{itemize}
%	\item Protected PFOB access, disable at time $t$, then Protected UPFOB access
%	\item Unlock Car 2 with temporary PFOB access: \flag{tempfobaccess\_*}
%\end{itemize}
%\fi

%\parhead{SR3} Observing the communications between a fob and a car while unlocking should not allow an attacker to unlock the car in the future.
%\iflong
%\begin{itemize}
%	\item Protected UPFOB access
%	\item Recorded 3 unlock transactions between a paired fob and a car
%	\item Unlock Car 3 with passive replay attacks: \flag{passiveunlock\_*}
%\end{itemize}
%\fi

%\parhead{SR4} Having an unpaired fob should not allow an attacker to unlock a car without the corresponding paired fob and pairing PIN.
%\iflong
%\begin{itemize}
%	\item Protected UPFOB access and leaked pairing PIN for Car 4
%	\item Unlock Car 4 with just the PIN: \flag{leakedpin\_*}
%	\item Protected PFOB, UPFOB access for Car 5
%	\item Extract the pairing PIN from Car 5: \flag{pinextract\_*}
%\end{itemize}
%\fi

%\parhead{SR5} A car owner should not be able to add new features to a fob that did not get packaged by the manufacturer.

%\parhead{SR6} Access to a feature packaged for one car should not allow an attacker to enable the same feature on another car.
%\iflong
%\begin{itemize}
%	\item Protected PFOB, UPFOB access for Car 5
%	\item Unlocking Car 5 is out of scope
%	\item Packaged Feature 2 for Cars 0-4
%	\item Enable Feature 2 for Car 5: \flag{feature2\_*} 
%\end{itemize}
%\fi

%See \Cref{sec:build}--\ref{sec:unlock} for a description of how our modified PARED protocol satisfies each SR. With the threat model defined, we now describe the cryptographic primitives which provide assurances of security in this scenario.

%\iflong
%\textit{NOTE: Car 0 contains no flags, but \emph{does} provide the unprotected images for the car and paired fob. %, as well as a protected unpaired fob, paired fob, car, and PIN.
%With these capabilities, we expect that attackers can analyze our implementation for vulnerabilities.}
%\fi